
\documentclass[promaster]{thesis-uestc}

\title{面向数据价值共享的激励机制设计与实现}{An Incentive Mechanism for Data Sharing}

\author{罗通}{Luo Tong}
\advisor{罗光春\chinesespace 教授}{Dr. Guangchun Luo}
\school{计算机科学与工程学院\chineseleftparenthesis网络空间安全学院\chineserightparenthesis}{School of Computer Science and Engineering}
\major{计算机技术}{Computer Technology}
\studentnumber{201722060929}

\begin{document}

\makecover

\begin{chineseabstract}
为了适应日益增长的宽带信号和非线性系统的工程应用,用于分析瞬态电磁散射问题的时域积分方程方法研究日趋活跃。本文以时域积分方程时间步进算法及其快速算法为研究课题,重点研究了时间步进算法的数值实现技术、后时稳定性问题以及两层平面波算法加速计算等,主要研究内容分为四部分。

……

\chinesekeyword{时域电磁散射,时域积分方程,时间步进算法,后时不稳定性,时域平面波算法}
\end{chineseabstract}

\begin{englishabstract}
With the widespread engineering applications ranging from broadband signals and non-linear systems, time-domain integral equations (TDIE) methods for analyzing transient electromagnetic scattering problems are becoming widely used nowadays. TDIE-based marching-on-in-time (MOT) scheme and its fast algorithm are researched in this dissertation, including the numerical techniques of MOT scheme, late-time stability of MOT scheme, and two-level PWTD-enhanced MOT scheme. The contents are divided into four parts shown as follows.

\englishkeyword{time-domain electromagnetic scattering, time-domain integral equation (TDIE), marching-on in-time (MOT) scheme, late-time instability, plane wave time-domain (PWTD) algorithm}
\end{englishabstract}

\thesistableofcontents

\thesischapterexordium

\section{研究工作的背景与意义}

计算电磁学方法\citing{wang1999sanwei, liuxf2006, zhu1973wulixue, chen2001hao, gu2012lao, feng997he}从时、频域角度划分可以分为频域方法与时域方法两大类。频域方法的研究开展较早,目前应用广泛的包括:矩量法(MOM)\citing{xiao2012yi,zhong1994zhong}及其快速算法多层快速多极子(MLFMA)\citing{clerc2010discrete}方法、有限元(FEM)\citing{wang1999sanwei,zhu1973wulixue}方法、自适应积分(AIM)\citing{gu2012lao}方法等,这些方法是目前计算电磁学商用软件\footnote{脚注序号“\ding{172},……,\ding{180}”的字体是“正文”,不是“上标”,序号与脚注内容文字之间空1个半角字符,脚注的段落格式为:单倍行距,段前空0磅,段后空0磅,悬挂缩进1.5字符;中文用宋体,字号为小五号,英文和数字用Times New Roman字体,字号为9磅;中英文混排时,所有标点符号(例如逗号“,”、括号“()”等)一律使用中文输入状态下的标点符号,但小数点采用英文状态下的样式“.”。}(例如:FEKO、Ansys 等)的核心算法。由文献\cite{feng997he,clerc2010discrete,xiao2012yi}可知

\section{国内外研究历史与现状}
时域积分方程方法的研究始于上世纪60 年代,C.L.Bennet 等学者针对导体目标的瞬态电磁散射问题提出了求解时域积分方程的时间步进(marching-on in-time, MOT)算法。

\section{本文的主要贡献与创新}
本论文以时域积分方程时间步进算法的数值实现技术、后时稳定性问题以及两层平面波加速算法为重点研究内容,主要创新点与贡献如下:

\section{本论文的结构安排}
本文的章节结构安排如下:

\section{本章小结}

\chapter{相关理论基础}
时域积分方程(TDIE)方法作为分析瞬态电磁波动现象最主要的数值算法之一,常用于求解均匀散射体和表面散射体的瞬态电磁散射问题。

\section{基本概念}
hello

\subsection{激励相容}
DSIC

\subsection{最优拍卖}
Ideal

\section{麦尔森引理}

\section{算法机制设计}

    \subsection{背包拍卖}

    \subsection{显示定理}

\section{税收最大化拍卖}

\section{简单近似拍卖}

\section{多变量环境}

\section{预算受限的机制设计}

\section{无金钱机制设计}

\section{本章小结}

\chapter{单变量环境下的机制}

\section{本章小结}

\chapter{多变量环境下的机制}

\section{本章小结}

\chapter{全文总结与展望}

\section{全文总结}
本文以时域积分方程方法为研究背景,主要对求解时域积分方程的时间步进算法以及两层平面波快速算法进行了研究。

\section{后续工作展望}
时域积分方程方法的研究近几年发展迅速,在本文研究工作的基础上,仍有以下方向值得进一步研究:

\thesisacknowledgement
在攻读博士学位期间,首先衷心感谢我的导师XXX教授


\nocite{*}
\thesisloadbibliography{reference}

%
% Uncomment following codes to load bibliography database with native
% \bibliography command.
%
% \nocite{*}
% \bibliographystyle{thesis-uestc}
% \bibliography{reference}
%

\thesisappendix

\chapter{中心极限定理的证明}

\section{高斯分布和伯努利实验}

\thesisloadaccomplish{publications}


\thesistranslationoriginal
\section{The OFDM Model of Multiple Carrier Waves}


\thesistranslationchinese

\section{基于多载波索引键控的正交频分多路复用系统模型}

\end{document}
