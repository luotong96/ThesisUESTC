
\documentclass[promaster]{thesis-uestc}

\title{面向数据价值共享的激励机制设计与实现}{An Incentive Mechanism for Data Sharing}

\author{罗通}{Luo Tong}
\advisor{罗光春\chinesespace 教授}{Dr. Guangchun Luo}
\school{计算机科学与工程学院\chineseleftparenthesis网络空间安全学院\chineserightparenthesis}{School of Computer Science and Engineering}
\major{计算机技术}{Computer Technology}
\studentnumber{201722060929}

\begin{document}

\makecover

\begin{chineseabstract}
为了适应日益增长的宽带信号和非线性系统的工程应用,用于分析瞬态电磁散射问题的时域积分方程方法研究日趋活跃。本文以时域积分方程时间步进算法及其快速算法为研究课题,重点研究了时间步进算法的数值实现技术、后时稳定性问题以及两层平面波算法加速计算等,主要研究内容分为四部分。

……

\chinesekeyword{时域电磁散射,时域积分方程,时间步进算法,后时不稳定性,时域平面波算法}
\end{chineseabstract}

\begin{englishabstract}
With the widespread engineering applications ranging from broadband signals and non-linear systems, time-domain integral equations (TDIE) methods for analyzing transient electromagnetic scattering problems are becoming widely used nowadays. TDIE-based marching-on-in-time (MOT) scheme and its fast algorithm are researched in this dissertation, including the numerical techniques of MOT scheme, late-time stability of MOT scheme, and two-level PWTD-enhanced MOT scheme. The contents are divided into four parts shown as follows.

\englishkeyword{time-domain electromagnetic scattering, time-domain integral equation (TDIE), marching-on in-time (MOT) scheme, late-time instability, plane wave time-domain (PWTD) algorithm}
\end{englishabstract}

\thesistableofcontents

\thesischapterexordium

\section{研究工作的背景与意义}

计算电磁学方法\citing{wang1999sanwei, liuxf2006, zhu1973wulixue, chen2001hao, gu2012lao, feng997he}从时、频域角度划分可以分为频域方法与时域方法两大类。频域方法的研究开展较早,目前应用广泛的包括:矩量法(MOM)\citing{xiao2012yi,zhong1994zhong}及其快速算法多层快速多极子(MLFMA)\citing{clerc2010discrete}方法、有限元(FEM)\citing{wang1999sanwei,zhu1973wulixue}方法、自适应积分(AIM)\citing{gu2012lao}方法等,这些方法是目前计算电磁学商用软件\footnote{脚注序号“\ding{172},……,\ding{180}”的字体是“正文”,不是“上标”,序号与脚注内容文字之间空1个半角字符,脚注的段落格式为:单倍行距,段前空0磅,段后空0磅,悬挂缩进1.5字符;中文用宋体,字号为小五号,英文和数字用Times New Roman字体,字号为9磅;中英文混排时,所有标点符号(例如逗号“,”、括号“()”等)一律使用中文输入状态下的标点符号,但小数点采用英文状态下的样式“.”。}(例如:FEKO、Ansys 等)的核心算法。由文献\cite{feng997he,clerc2010discrete,xiao2012yi}可知

\section{国内外研究历史与现状}
时域积分方程方法的研究始于上世纪60 年代,C.L.Bennet 等学者针对导体目标的瞬态电磁散射问题提出了求解时域积分方程的时间步进(marching-on in-time, MOT)算法。

\section{本文的主要贡献与创新}
本论文以时域积分方程时间步进算法的数值实现技术、后时稳定性问题以及两层平面波加速算法为重点研究内容,主要创新点与贡献如下:

\section{本论文的结构安排}
本文的章节结构安排如下:

\section{本章小结}

\chapter{相关理论基础}
时域积分方程(TDIE)方法作为分析瞬态电磁波动现象最主要的数值算法之一,常用于求解均匀散射体和表面散射体的瞬态电磁散射问题。

\section{基本概念}
hello

\subsection{激励相容}
DSIC

\subsection{最优拍卖}
Ideal

\section{麦尔森引理}

\section{算法机制设计}

    \subsection{背包拍卖}

    \subsection{显示定理}

\section{税收最大化拍卖}

\section{简单近似拍卖}

\section{多变量环境}

\section{预算受限的机制设计}

\section{无金钱机制设计}

\section{本章小结}

\chapter{简单可并行计算下的机制}

\section{引言}

\section{基本假设}
在简单可并行计算下的机制建立之前,首先给出相关合理假设。

假设1:参与者是有限理性的个体,总是执行理性条件下最大化各自效用函数的策略,且不存在共谋

假设2:除参与者对原位计算任务代价的内心估值为私密知识以外,其余均为公共知识,为平台方和其它参与者所知。
\section{基础模型定义}
机制中以虚拟代币作为货币。机制中有$n$个参与者$\mathbf{agent}$可以进入原位计算任务$T$的竞拍环节。原位计算任务$T$共需要$datademand$单位的计算。对于$1\leq i\leq n$,参与者$agent_i$可以针对任务书$T$提供$datacount_i$单位的原位计算任务。
在机制中,平台方会给予胜出参与者每单位计算任务$CENT$个货币作为奖赏。然而
参与者$agent_i$拥有私密值$0 \leq val_i = INCENT-cost_i$,$INCENT$是平台方给予胜出者的每单位计算任务的固定奖赏,由平台方或者原位计算任务发起者确定。$0 \leq cost_i \leq INCENT$,否则,参与者$agent_i$应当退出此次竞拍来避免自己受损。 于是,$val_i$表示$agent_i$对购买每单位原位计算任务带来的真实价值。其$val_i$只对参与者$agent_i$可见,平台方及其余参与者不可见。在机制中,每个参与者会提交自己的标的$b_i$,形成标的向量$\mathbf{bid}$。分配函数$\mathbf{allocation}(\mathbf{bid})$也是一个向量,表示了机制对参与者分配的原位计算任务数量。$agent_i$的效用函数定义为准线性函数$utility_i = val_i*allocation_i-payment_i$。总社会福利$\mathbf{SW}$定义为$\sum_{i=1}^n{val_i*allocation_i}$,其是所有参与者的效用函数与平台方收益的总和。即:$\mathbf{SW} = \sum_{i=1}^n{utility_i}+\sum_{i=1}^{n}{payment_i}$

\section{基础模型求解}
本文的目标是设计一种参与者具有占优策略的拍卖机制。由显示定理可知,对于任意一个参与者总是具有占优策略的机制$M$,总是存在一个等价的直接显示(direct-revelation)占优策略激励相容(DSIC)机制$M'$。因此,本文在DSIC机制的范围中寻找合适的解。而在前述基础模型定义中,参与者的私密值仅为其对购买原位计算任务的收益真实估值$val_i$,且效用函数为准线性函数。这属于单变量环境,需要麦尔森引理来主导机制设计的流程。

假设所有参与者已经真实地表达原位计算任务的估值,并给出标的,即$\mathbf{bid} = \mathbf{val}$。 此时对于平台方来说,社会福利最大化等价于如下最优化问题。

\begin{align}
    \label{jichumoxing}\max_{\vec{allocation}} \quad&\sum_{i=1}^{n}{bid_i*allocation_i}\\
    s.t                     \quad& 0\leq allocation_i\leq datacount_i,\quad allocation_i = {0,1,2...}\notag\\
                            \quad& \sum_{i=1}^{n}{allocation_i}\leq datademand\notag 
\end{align}

给出贪心算法:\footnote{本文中数组下标均从1开始计数}
\footnote{可以通过first和second来分别指代二元组的第一、二关键字}

\begin{algorithm}[H] 
    \KwIn{$\mathbf{datacount},\mathbf{bid},datademand$}
    \KwOut{$\mathbf{allocation}$}
    $i \leftarrow 1$\;
    $total \leftarrow datademand $\;
    对二元组序列$list = \{(bid_j,j)|1 \leq j \leq n\}$以第一关键字进行降序排列,第一关键字相同的情况下,以第二关键字升序排列\;
    \While{total > 0}{
        \If{i > n}{
            break\;
        }
        $temp \leftarrow \min(total,\quad datacount_{list[i].second})$\;
        $total \leftarrow total - temp$\;
        $allocation_i \leftarrow temp$\;
        $i \leftarrow i + 1$\;
    }
\caption{贪心算法求解基础模型}
\label{tanxin}
\end{algorithm}

\begin{theorem}
算法\ref{tanxin}是问题\ref{jichumoxing}的解。
\end{theorem}

\begin{proof}
    算法\ref{tanxin}的思想是在供应$datademand$有限的情况下,优先满足代价估值$bid_i$更大的参与者的需求。假设存在一个更优解$S^1(SW^1 > SW)$与该思想相悖。即存在$1 \leq i \neq j \leq n,bid_i > bid_j$, 使得$allocation_i < datacount_i \text{, } allocation_j > 0$。不妨将分配给参与者$agent_j$的$temp = min(datacount_i - allocation_i,allocation_j)$单位计算任务重新分配给参与者$agent_i$,得到解$S^2$,此时新的社会福利$SW^2= SW^1- temp*bid_j + temp*bid_i=SW^1+temp*(bid_i-bid_j)$,由前述条件可知,$temp > 0$且$bid_i - bid_j > 0$。易得,$SW^1<SW^2 $。继续对$S^2$执行以上流程,直到前提条件不满足。此时,解$S^n$优先满足代价估值更大的参与者的需求。由贪心算法可知$SW^n = SW$,则$SW^1<SW^2...<SW^n=SW$,这与假设不符。故不存在上述更优解。
\end{proof}

分配函数已经确定,为了应用麦尔森引理确定价格函数,需要首先分析分配函数关于标的$bid_i$的单调性。

\begin{theorem}
    对于任意参与者$agent_i$以及任意$\mathbf{bid_{-i}}$\footnote{表示除$bid_i$分量以外的总标的向量},算法\ref{tanxin}所示分配规则$allocation_i(z,\mathbf{bid_{-i}})$是关于$agent_i$的标的$z$的单调不减函数。
\end{theorem}

\begin{proof}
   在保持其余标的不变的情况条件下,若$agent_i$的标的$z$变大,则其在算法\ref{tanxin}中list序列中的排位只会更加靠前,从而获得更高优先级的分配权力。而算法的不确定性已经由第二关键字排序消除。
\end{proof}

$allocation_i(z,\mathbf{bid_{-i}})$的函数图像如下所示。

\begin{figure}
    
\end{figure}

根据麦尔森引理,阴影部分面积即为$agent_i$向平台方支付的报酬。现给出支付函数计算规则,如算法\ref{tanxin_zhifu}所示

\begin{algorithm}[H]
    \KwIn{$\mathbf{datacount},\mathbf{bid},datademand$}
    \KwOut{$\mathbf{payment}$}
    运行算法\ref{tanxin},获得list\;
    \tcp{根据标的排序求count的各前缀和}
    prefix 置为初始值为0的数组\;
    \For{$i = 1 \KwTo n$}
    {
        \If{i > 1}{
            $prefix[i] \leftarrow prefix[i-1]$\;
        }
        $prefix[i]  += datacount_{list[i].second}$\;
    }
    \tcp{求每个参与者的支付$payment_i$}
    \For{$i = 1 \KwTo n$}
    {
        $payment_{list[i].second} \leftarrow 0$\;
        $ind \leftarrow i + 1$\tcp*{ind存储遍历下标}\;
        \tcp{y存储当前分配值}
        \eIf{$i = 1$}
        {
            $y \leftarrow min(datademand,datacount_{list[i].second})$\;
        }{
        $y \leftarrow min(max(datademand - prefix[i-1],0),datacount_{list[i].second})$\;
        }
        \While{$ind \leq n$ 且 $y > 0$}
        {
            $newy \leftarrow min(max(datademand - prefix[ind]+ datacount_{list[i].second},0),datacount_{list[i].second})$\;
            $payment_{list[i].second} += (y - newy)*list[ind].first$\;
            $y = newy$\;
            $ind += 1$\;
        }
    }
    \caption{贪心算法求解基础模型的支付流程}
    \label{tanxin_zhifu}
\end{algorithm}

\section{本章小结}

\chapter{多变量环境下的机制}

\section{本章小结}

\chapter{全文总结与展望}

\section{全文总结}
本文以时域积分方程方法为研究背景,主要对求解时域积分方程的时间步进算法以及两层平面波快速算法进行了研究。

\section{后续工作展望}
时域积分方程方法的研究近几年发展迅速,在本文研究工作的基础上,仍有以下方向值得进一步研究:

\thesisacknowledgement
在攻读博士学位期间,首先衷心感谢我的导师XXX教授


\nocite{*}
\thesisloadbibliography{reference}

%
% Uncomment following codes to load bibliography database with native
% \bibliography command.
%
% \nocite{*}
% \bibliographystyle{thesis-uestc}
% \bibliography{reference}
%

\thesisappendix

\chapter{中心极限定理的证明}

\section{高斯分布和伯努利实验}

\thesisloadaccomplish{publications}


\thesistranslationoriginal
\section{The OFDM Model of Multiple Carrier Waves}


\thesistranslationchinese

\section{基于多载波索引键控的正交频分多路复用系统模型}

\end{document}
